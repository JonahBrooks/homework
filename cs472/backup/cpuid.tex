\documentclass[letterpaper,10pt,titlepage]{article}

\usepackage{graphicx}                                        
\usepackage{amssymb}                                         
\usepackage{amsmath}                                         
\usepackage{amsthm}                                          

\usepackage{alltt}                                           
\usepackage{float}
\usepackage{color}
\usepackage{url}

\usepackage{balance}
\usepackage[TABBOTCAP, tight]{subfigure}
\usepackage{enumitem}
\usepackage{pstricks, pst-node}

\usepackage{geometry}
\geometry{textheight=8.5in, textwidth=6in}

%random comment

\newcommand{\cred}[1]{{\color{red}#1}}
\newcommand{\cblue}[1]{{\color{blue}#1}}

\usepackage{hyperref}
\usepackage{geometry}

\def\name{Jonah Brooks}

%pull in the necessary preamble matter for pygments output
\input{pygments.tex}

%% The following metadata will show up in the PDF properties
\hypersetup{
  colorlinks = true,
  urlcolor = black,
  pdfauthor = {\name},
  pdfkeywords = {cs472 ``computer architecture'' cpuid},
  pdftitle = {CS 472 Homework 1: CPUID},
  pdfsubject = {CS 472 Homework 1},
  pdfpagemode = UseNone
}

\begin{document}

\section{Questions}

\subsection{ Describe the difference between architecture and organization. }
% 2: Describe the difference between architecture and organization.
Architecture comprises the methodology, software, and logical structure of a CPU.
It includes things like the methodological decision between RISC vs CISC,
the specific assembly language used for that CPU,
and the specific (and generally secret) microarchitecture beneath the assembly language.

Organization has more to do with the hardware.
It encompasses the physical layout of the transistors, memory, etc. 

\subsection{Describe the concept of endianness. What common platforms use what endianness?}
% 3: Describe the concept of endianness. What common platforms use what endianness?
Endianness has to do with how sequential bytes of a single value are stored in memory.
Big-endian, as used in SPARC, PowerPC, and others, stores the most significant byte first 
(that is, in the lowest memory address).  
Little-endian, as used in x86, ARM, and others, stores the most significant byte last 
(that is, in the highest memory address).

\subsection{Give the IEEE 754 floating point format for both single and double precision.}
% 4: Give the IEEE 754 floating point format for both single and double precision.
Both single and double precision IEEE 754 floating point formats have 1 sign bit at the beginning.
From there, single has 8 exponent bits, and double has 11 exponent bits.
Finally, single has 23 bits for the fractional portion, while double has 52 bits.


\subsection{Describe the concept of the memory hierarchy. What levels of the hierarchy are present on flip.engr.oregonstate.edu?}
% 5: Describe the concept of the memory hierarchy. What levels of the hierarchy are present on flip.engr.oregonstate.edu?
Memory hierarchy is the structural layout of memory accessible by the CPU.
This typically contains one or more levels of cache, main memory, and physical drives attached to the computer.
Flip has two level 1 caches, one for instructions and one for data,
a level 2 unified cache, RAM, and hard drive space.

\subsection{What streaming SIMD instruction levels are present on flip.engr.oregonstate.edu?}
% 6: What streaming SIMD instruction levels are present on flip.engr.oregonstate.edu?
SSE, SSE2, SSE3, SSSE3, and SSE4.1. 

\section{Source Code}
%input the pygmentized output of foo.c, using a (hopefully) unique name
%this file only exists at compile time. Feel free to change that.
\input{__cpuid.c.tex}
\end{document}
