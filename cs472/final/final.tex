%Jonah Brooks, CS472 Final Paper
\documentclass{article}
\usepackage{url}
\usepackage[backend=bibtex]{biblatex}
\bibliography{final.bib}
\begin{document}

\begin{flushright}
Jonah Brooks

CS472

Final Report
\end{flushright}

\section*{1: Introduction and History}

The PowerPC architecture was originally designed by Apple, IBM, and Motorola
(sometimes collectively referred to as AIM) in 1990. The main goal was to
optimize the existing POWER architecture while reducing its cost. This
endeavor involved many changes from the original POWER architecture, including
support for out of order execution, more streamlined compound instructions, as
well as numerous differences in branch handling. \cite{1}

\section*{2: Instruction set design}

The original POWER architecture upon which PowerPC was based was a RISC
architecture. In order to maintain support for software designed for those
systems, the PowerPC also adopted a RISC architecture of its own. The
inclusion of somewhat more complex compound instructions, such as an
instruction to calculate an effective address during a load and store, added a
few CISC-like elements, but the architecture as a whole remained RISC based.
\cite{1}

This is partially shown in the fact that PowerPC systems use only a register
plus displacement addressing mode for instructions and data \cite{2}, as well as
an index based addressing mode for data \cite{4}. 
Even other RISC architectures typically allow for more addressing modes, such as ARM's 10 different modes. \cite{3}

Each PowerPC instruction, including any operands used in the instruction, fits
into a single word. This allows the fetch and decoding steps to be
streamlined, as every instruction will occupy the same easily accessible size.
\cite{5} This is largely the same way ARM does it, with each instruction taking
exactly one word. \cite{3}


\section*{3: Datapath design}

PowerPC has 32 general purpose registers, 32 floating point registers, as well
as numerous special purpose registers. \cite{6} This is markedly different from
ARM, which has only 16 general purpose registers visible to the programmer. \cite{3}

The various PowerPC chips handle pipelining slightly differently, ranging from
4 to 7 stages. \cite{7} The PowerPC 620, for instance, has 5 primary stages of
pipelining: Fetch, Dispatch, Execute, Complete, and Writeback, \cite{8} with some
of those stages having their own smaller pipelines depending on data and
instruction types. The ARM architecture on the
other hand has 9 stages \cite{9}, with Fetch and
Execute broken up into 3 and 4 stages each, respectively.

\section*{4: Memory subsystem}

Virtual memory is handled in PowerPC by hashing virtual memory addresses
created using Process ID and other MMU registers, and
concatenating the hash with the contents of a special purpose register
(specifically the SDR1 register). \cite{10} Some implementations use the 8th MMU
assist register (MAS7) to allow for virtual addresses of greater than 32 bits. 

Cache levels and size vary somewhat by implementation. The IBM PowerPC 970FX,
for instance, has a 32KB L1 data cache, a 64KB L1 instruction cache, and a
512KB L2 cache. \cite{11} The cache levels vary between ARM implementations as
well, but are generally around this level. The ARM based Cortex-8, for
instance, has 16KB or 32KB for both L1 data and L1 instruction cache, and 64KB
to 2MB of L2 cache. \cite{12}

\section*{5: Other HPC focused characteristics in this system}

PowerPC based systems have been a staple at the top of the worlds most
powerful supercomputers for many years. \cite{13} The Blue Gene line of
supercomputing processors, for instance, are based on the PowerPC architecture
and have hardware support for transactional memory in addition to specialized
FPUs. \cite{14}

\section*{6: An explanation of how the features of this system might boost performance.}

PowerPC may not have much of a hold in the consumer electronics industry at the moment,
but there are aspects of its design that can provide a nice boost to performance in some
use cases. As mentioned earlier, PowerPC has an interesting set of instructions, many of which
are capable of doing a lot in a single CPU cycle. For projects in which those instructions are of use,
PowerPC can do more per instruction than similar RISC systems. Unfortunately, PowerPC systems typically
have a lower instructions per cycle ratio than other similar systems, so the gain is largely lost unless
those optimized instructions are a large portion of the code.

Another way in which PowerPC can offer a boost that many other architectures cannot is in its extensibility.
This, as well as how useful it can be, is shown in the Blue Gene systems that extend PowerPC with transactional memory
support and enhanced floating point processing. This allows PowerPC based chips to cater especially well to certain
use cases.  

\section*{Bibliography}
\printbibliography
\end{document}
